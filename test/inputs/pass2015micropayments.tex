\section{Micropayments for Decentralized Currencies }
\bibentry{pass2015micropayments}

\textbf{Abstract:} 
Electronic financial transactions in the US, even those enabled by Bitcoin, have relatively high transaction costs. As a result, it becomes infeasible to make micropayments, i.e. payments that are pennies or fractions of a penny. To circumvent the cost of recording all transactions, Wheeler (1996) and Rivest (1997) suggested the notion of a probabilistic payment, that is, one implements payments that have expected value on the order of micro pennies by running an appropriately biased lottery for a larger payment. While there have been quite a few proposed solutions to such lottery-based micropayment schemes, all these solutions rely on a trusted third party to coordinate the transactions; furthermore, to implement these systems in today’s economy would require a a global change to how either banks or electronic payment companies (e.g., Visa and Mastercard) handle transactions. We put forth a new lottery-based micropayment scheme for any ledger-based transaction system, that can be used today without any change to the current infrastructure. We implement our scheme in a sample web application and show how a single server can handle thousands of micropayment requests per second. We analyze how the scheme can work at Internet scale.
