\section{Bitcoin Blockchain Dynamics: the Selfish-Mine Strategy in the Presence of Propagation Delay }
\bibentry{gobel2015simulation}

\textbf{Abstract:} 
In the context of the ‘selfish-mine’ strategy pro- posed by Eyal and Sirer we study the effect of propagation delay on the evolution of the Bitcoin blockchain. First we u se a simplified Markov model that tracks the contrasting states of belief about the blockchain of a small pool of miners and the ‘rest of the community’ to establish that the use of block-hi ding strategies such as selfish-mine causes the rate of product ion of orphan blocks to increase. Then we use a spatial Poisson proc ess model to study values of Eyal and Sirer’s parameter γ which denotes the proportion of the honest community that mine on a previously-secret block released by the pool in response to the mining of a block by the honest community. Finally we use discrete-event simulation to study the behaviour of a netwo rk of Bitcoin miners a proportion of which is colluding in usin g the selfish-mine strategy under the assumption that there i s a propagation delay in the communication of information betw een miners
