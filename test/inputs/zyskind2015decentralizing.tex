\section{Decentralizing Privacy: Using Blockchain to Protect Personal Data }
\bibentry{zyskind2015decentralizing}

\textbf{Abstract:} 
The recent increase in reported incidents of surveillance and security breaches compromising users’ privacy call into question the current model, in which third-parties collect and control massive amounts of personal data. Bitcoin has demonstrated in the financial space that trusted, auditable computing is possibleusing a decentralized network of peers accompanied by a public ledger.  In  this  paper,  we  describe  a  decentralized  personal  data management  system  that  ensures  users  own  and  control  their data.  We  implement  a  protocol  that  turns  a  blockchain  into  an automated access-control manager that does not require trust in a third party. Unlike Bitcoin, transactions in our system are not strictly  financial  –  they  are  used  to  carry  instructions,  such  as storing,  querying  and  sharing  data.  Finally,  we  discuss  possible future  extensions  to  blockchains  that  could  harness  them  into  a well-rounded solution for trusted computing problems in society.
